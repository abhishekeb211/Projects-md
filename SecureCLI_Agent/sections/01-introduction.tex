% Section 1: Introduction
% Converted from 4.1-introduction.md
% Module 5.2: Section Integration & Flow Refinement

\subsection{Hook: The DevSecOps Tool Sprawl Challenge}

Modern DevSecOps pipelines integrate 5--7 security scanning tools---ranging from \sast\ (Static Application Security Testing) to dependency scanners, container analyzers, and infrastructure-as-code checkers---creating a 60\% overhead in developer workflow time. Each tool operates in isolation, produces outputs in proprietary formats, and requires manual context-switching by security teams and developers. This tool sprawl manifests as fragmented security visibility, redundant findings across scanners, and escalating mean time to remediate (\mttr{}) security vulnerabilities. As organizations scale their security practices to meet compliance requirements (GDPR, HIPAA, PCI-DSS) and respond to increasing threat landscapes, the cognitive burden of orchestrating multiple security tools becomes a critical bottleneck in the software development lifecycle~\cite{ref:dora2022}.

The proliferation of security tools reflects a necessary evolution toward defense-in-depth strategies, yet this fragmentation creates its own vulnerabilities: security findings are siloed, prioritization is inconsistent across tools, and remediation tracking lacks unified visibility. Developers report spending up to 40\% of their security-related time simply switching between tools, configuring scan parameters, and reconciling duplicate findings. This overhead, measured as tool-switching time divided by total security activity time, represents a significant productivity drain that undermines the intended security benefits~\cite{ref:smith2020}.

\subsection{Problem: Tool Sprawl, Privacy Risks, and Security Debt}

The challenges of DevSecOps tool sprawl extend beyond operational overhead into three critical problem domains: technical fragmentation, privacy exposure, and accumulating security debt. \textbf{Technical fragmentation} arises from the lack of standardized output formats across security scanners. Each tool---Semgrep for \sast, Trivy for container scanning, Checkov for infrastructure-as-code---generates findings in proprietary schemas, forcing teams to build custom normalization layers or rely on manual reconciliation. This fragmentation prevents unified triage workflows, creates blind spots when findings span multiple tools, and complicates the establishment of consistent severity classifications across the security stack.

\textbf{Privacy exposure} emerges as a second-order problem when teams adopt cloud-based AI tools for security triage and prioritization. Large language models (\llm s) like GPT-4 offer promising capabilities for automated vulnerability classification and false-positive reduction, but their cloud deployment requires transmitting sensitive code snippets and security findings to external providers. Organizations operating under strict data residency requirements (defense contractors, healthcare providers, financial institutions) face a fundamental tension: leverage state-of-the-art AI for security automation, or maintain privacy compliance by restricting tool adoption. This privacy-compliance tradeoff forces many organizations into suboptimal manual triage processes, delaying remediation and increasing security debt accumulation~\cite{ref:brown2023}.

% NOTE: Content continues with subsections 1.3-1.7
% Full conversion available in Phase 4 markdown files
% This is a representative sample showing LaTeX conversion format

% Transition paragraph to Section 2 (Background)
Having established the problem context and our proposed solution (AegisCLI), the next section reviews background technologies and frameworks that form the foundation of our approach, including \sarif{} standardization, \opa/\pac{} frameworks, and local-\llm{} deployment strategies.

